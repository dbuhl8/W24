For a threshold of $10^{-4}$ the code took 3 iterations to complete and returned a difference of $.52632 \cdot 10^{-5}$. For a threshold of $10^{-8}$ the code took 5 iterations to complete and returned a difference of $.81295 \cdot 10^{-8}$. For a threshold of $10^{-12}$ the code took 8 iterations to complete and returned a difference of $.82023 \cdot 10^{-12}$. For a threshold of $10^{-16}$ the code took 11 iterations to complete and returned a difference of $.00000 \cdot 10^{00}$. The last output which is correlated to the lowest threshold, it makes sense that the difference is registered at 0. This would be because double precision variables in fortran go out to only 16 decimal places. Thus, when the threshold is so low, fortran can no longer distinguish the variables. 
