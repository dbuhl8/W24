\documentclass{article}
\usepackage{graphicx} % Required for inserting images
\usepackage[margin=1in]{geometry}
\usepackage{amsmath}
\usepackage{amsthm}
\usepackage{amssymb}
\usepackage{amsfonts}
\usepackage{verbatim}
\usepackage{xcolor}

\title{Homework 4: Report}
\author{Dante Buhl}
\date{Feb. $26^{th}$ 2024}


\DeclareMathOperator{\cond}{cond}
\DeclareMathOperator{\vecspan}{span}

\begin{document}

\newcommand{\bs}[1]{\boldsymbol{#1}}
\newcommand{\bmp}[1]{\begin{minipage}{#1\textwidth}}
\newcommand{\emp}{\end{minipage}}
\newcommand{\R}{\mathbb{R}}
\newcommand{\C}{\mathbb{C}}
\newcommand{\N}{\mathcal{N}}
\newcommand{\I}{\mathrm{I}}
\newcommand{\K}{\bs{\mathrm{K}}}
\newcommand{\m}{\bs{\mu}_*}
\newcommand{\s}{\bs{\Sigma}_*}
\newcommand{\dt}{\Delta t}
\newcommand{\tr}[1]{\text{Tr}(#1)}
\newcommand{\Tr}[1]{\text{Tr}(#1)}

\maketitle


\setcounter{section}{1}

\section{Cholesky Solution of the least-squares problem}
\begin{enumerate}
\item How does the Cholesky Decomposition Work?

The cholesky decomposition is merely a special case of the LU factorization method. It is designed to create a matrix $L$ such that $LL* = A$. That is, we find that $L$ is (with abuse of notation) a "square-root" of $A$ in matrix form. The algorithm is very simple, we start with the $l_{11}$ component of $L$. We have that $l_{11} = \sqrt{a_{11}}$. We proceed just like LU factorization except that the symmetry of $A$ allows us to perform half of the computations necessary (this is due to the symmetry of the problem). You can throw away half of the matrix $A$ (above the diagonal) upon entry and L wouldn't change. 

\item The results of fitting a third degree polynomial are a frobenius norm of error at $||E||_F = 0.19127$. The coefficients build the following polynomial, 
    \[
        f(x) = 0.578 + 4.666x - 10.935 x^2 + 7.514x^3
    \]


\end{enumerate}

\section{QR Solution of the least-squares problem}
\begin{enumerate}
\item 

\end{enumerate}

\section{Theory Problems}
\begin{enumerate}

\item %start of problem 1
Show that is $P$ is an orthogonal projector, then $\I - 2P$ is unitary. 

\begin{proof}
    We begin with the definition of a unitary matrix. We have a unitary matrix $Q$ is a matrix such that $Q^*Q = \I$. We now look for this quality in $(\I - 2P)$. 
    \[
        (\I - 2P)^*(\I - 2P) = (\I - 2P^T)(\I - 2P) = (\I - 2P)(\I - 2P)
    \]
    The above quality $P^T = P$ is from the fact that $P$ is an orthogonal projector. We also have the quality that $P^2 = P$.
    \[
        (\I - 2P)(\I - 2P) = \I(\I - 2P) - 2P(\I - 2P) = \I - 2P - 2P + 4P^2
    \]
    \[
        = \I - 4P +4P = \I
    \]  
    We have recovered the condition for unitary matrices. We can therefore declare $(\I - 2P)$ a unitary matrix. 
\end{proof}


\item % start of problem 2
Let $P \in \R^{m \times m}$ be a nonzero projector. 
\begin{enumerate}
    \item Show that $||P||_2 \ge 1$, with equality if and only if $P$ is an orthogonal projector. 
    \begin{proof}
        (i) without equality
        
        We begin by looking at a projector $P$ which transforms a vector onto the span of another unit vector $v$. That is $P: \R^m \to \vecspan(v)$. We look at the case of the definition of the two-norm for matrices. 
        \[
            ||P||_2 = \sup_{x} \frac{||Px||_2}{||x||_2}
        \]
        We now take the case $x = v$. 
        \[
            ||P||_2 = \sup_{x} \frac{||Px||_2}{||x||_2} \ge \frac{||Pv||_2}{||v||_2}
        \]
        We notice that $v \in \vecspan(v)$, so the transformation $P$ is the identity for $v$. So, 
        \[
            ||P||_2 \ge \frac{||Pv||_2}{||v||_2} = \frac{||v||_2}{||v||_2} = 1
        \]
        \[
            ||P||_2 \ge 1
        \]

        (ii) with equality ($\implies$)
    
        We now look more closely at the definition of $P$ and the singular value decomposition of $P$. If $P$ is orthogonal we have that $P = vv^T$ for some unit vector $v \in \R^m$. We also have by the singular value theorem, that a singular value will satisfy the following property, 
        \[
            Pv_i = \sigma_iu_i, \quad P = U\Sigma V^T
        \]
        Where we have that $v_i, u_i$ are the i-th column vectors of $V, U$ respectively, and $\sigma_i$ is the i-th diagonal element of $\Sigma$. Note that $U, V$ are unitary and as a consequence its column vectors are orthogonal and have two-norm of 1. We look at some $P = xx^T$ for a unit vector $x$. 
        \[
            Pv_i = \sigma_iu_i \to xx^Tv = \sigma_iu_i
        \]
        \[
            (x, v_i) x = \sigma_i u_i
        \]
        We notice that vectors $x, u_i$ are related by scalars as a consequence of this definition of $P$. Therefore $x$ and $u_i$ must be colinear, however this is not guaranteeed by our assumptions. We have a few consequences and cases. Either $x$ and $u_i$ are colinear, or they are not. We look at the case they are colinear, 
        \[
            x = \alpha u_i
        \]
        We then must have that $\alpha = \frac{\sigma_i}{(x, v_i)}$. We then look at one of our prior assumptions. We have most importantly that $||x||_2 = ||v_i||_2 = ||u_i||_2 = 1$. 
        \[
            ||x||_2 = |\alpha|||u_i||_2 = 1, \implies |\alpha| = 1
        \]
        \[
            \sigma_i = \pm (x, v_i)
        \]
        We need two more things. First, that singular values cannot be negative by definition. Second, we have that if both $x$, $v_i$ are unit vectors, we cannot have that their inner product is greater than 1. Another way of expressing this is the geometrical interpretation that the dot product is $x \cdot v_i = (x, v) = ||x||_2||v_i||_2\cos(\theta)$. If $||x||_2 = ||v_i||_2 = 1$, $(x, v_i) = \cos(\theta) \le 1$. Finally, (and I mean it this time), we look at the case where $Px = x$ we have that since $x$ is a unit vector we recover an eigenvalue (in this case also a singular value) of $P$. Thereby we officially have, 
        \[
            0 < \sigma_i \le 1, \sigma_1 = 1
        \] 
        We use a proof from a different homework problem (or maybe from the lecture note, I can't remember where) relating $||A||_2 = \sigma_1$, to show, 
        \[
            ||P||_2 = \sigma_1 = 1
        \]

        ($\impliedby$) If $||P||_2 = 1$, then $P$ is an orthogonal projector (i.e. $P^T = P$)

        We begin by looking at the definition of the two norm for matrices. We have, 
        \[
            ||P||_2 = \sup_{x} \frac{||Px||_2}{||x||_2} = 1
        \]
        Therefore the case exists such that we find, 
        \[
            ||Px||_2 = ||x||_2
        \]
        We then introduce the fact that a two norm of a vector is the square root of the inner product of that vector with itself. That is $||v||_2 = \sqrt{(v, v)}$. Thus we have, 
        \[
            \sqrt{x^TP^TPx} = \sqrt{x^Tx}
        \]
        \[
            x^TP^TPx = x^Tx
        \]
        \[
            P^TPx = x = PPx, \quad \text{by definition of a projector}
        \]
        \[
            P^TPx = PPx, \implies P^Ty = Py
        \]
        This implies that for any vector in the span of $v$ (We take $P: \R^m \to \vecspan(v)$), that $P^T = P$. Moreover we have that for any vector $x$, $P^TP = P^2$. If we take $P$ to be invertible, we have that $P^T = P$. Therefore, we have that $P$ is idempotent and symmetric, making it an orthogonal projector. 
        
    \end{proof}

    \item Show that if $P$ is an orthogonal projector, then P is semi-positive deinite with its eigenvalues either zero or 1. 
        \begin{proof}
            We look at the vector product definition of $P$. $P = xx^T$ for a unit vector $x$. 
            \[
                (v, Pv) = v^Txx^Tv = (v, x)(x, v) = (x, v)^2 \ge 0
            \]
            Since our choice of $v$ was arbitary we have that $P$ is semi-positive definite. 
            
            Next we look at the eigenvalues of $P$. Say that we have an arbitary eigenvalue-eigenvector pair ($\lambda, v$) for $P$ such that $v \neq \vec{0}$ (obviously). We have,
            \[
                Pv = \lambda v
            \]
            \[
                Pv = xx^Tv = (x, v)x = \lambda v
            \]
            Notice that $(x, v)$ and $\lambda$ are scalars. This implies that $x$ and $v$ are colinear but this was not an assumption made. Therefore we are left with two cases: $v$ and $x$ are colinear, or $v$ and $x$ are orthogonal. Let's look at the first case, $v = \alpha x$.
            \[
                (x, v)x = \alpha x = \alpha\lambda x
            \] 
            \[
                x = \lambda x \implies \lambda = 1
            \]
            We find that for all vectors colinear to $x$ are eigenvectors with eigenvalue 1. We look at the other case. If $x$ and $v$ are orthogonal we have $(x, v) = 0$. 
            \[
                \vec{0} = \lambda v \implies \lambda = 0
            \]
            Therefore all vectors orthogonal to $x$ will be eigenvectors with $\lambda = 0$. 
        \end{proof}
\end{enumerate}

\item   % start of problem 3
Let $A \in \R^{m \times n}$ with $m \ge n$, and let $A = \hat{Q}\hat{R}$ be a reduced $QR$ factorization. 

\begin{enumerate}
\item Show that $A$ has rank $n$ if and only if all the diagonal entries of $\hat{R}$ are nonzero.
    
\begin{proof}
    
    ($\implies$) $A$ is rank $n$ if all of the diagonal entries of $\hat{R}$ are nonzero. 
    
    Let us look at the reduced QR factorization of $A$ such a that all diagonal entries of $\hat{R}$ are nonzero. 
    \[
        A = \hat{Q}\hat{R} = \left[q_1 \Big| \cdots \Big| q_n\right]\hat{R}
    \]
    We have by construction of a QR factorization that the matrix $\hat{Q}$ is composed of orthogonal unit column vectors $q_i$. Let us now look at the column vectors of $A$. 
    \[
        a_i = r_{1i}q_1 + \cdots + r_{ii}q_i
    \]
    Notice that since all diagonal elements of $\hat{R}$ are nonzero that we have that each $a_i$ is immediately distinguished from $a_{i-1}$ by the inclusion of the vector $q_i$. Let us start a small induction proof. Take the base case to demonstrate that $a_1$ is linearly indepentent from $a_2$. By contradiction suppose that $a_1, a_2$ are linearly dependent. 
    \[
        0 = c_1a_1 + c_2a_2 = c_{1}r_{11}q_1 + c_2r_{12}q_1 + c_2r_{22}q_2 = (c_1r_11 + c_2r_{12})q_1 + c_2r_{22}q_2
    \]
    \[
        0 = d_1q_1 + d_2q_2
    \]
    However, since the vectors $q_i$ are linearly independent, we require that $d_1 = d_2 = 0$ since the vectors $q_1, q_2$ are linearly independent. Immediately we notice that $c_2$ must equal zero since $r_{22} \neq 0$. Therefore for the two to be linearly dependent we must have that $c_1 \neq 0$. A contradiction is reached, since $d_1 = 0 = c_1r_{11} + 0 \implies r_{11} = 0$. Thus we have that $a_1, a_2$ are linearly independent. 
    
    Next we look at the inductive step. Take $a_1, \cdots, a_k$ to be linearly independent. Let us look at the set $a_1, \cdots, a_{k+1}$. We have evidently, that, 
    \[
        c_1a_1 + \cdots + c_ka_k = d_1q_1 + \cdots d_kq_k
    \]
    Such that $d_1q_1 + \cdots + d_kq_k = 0$ if and only if $d_1 = \cdots = d_k = 0 = c_1 = \cdots = c_k$. Let us now add $c_{k+1}a_{k+1}$ and look at the linear dependence. 
    \[
        c_1a_1 + \cdots + c_ka_k + c_{k+1}a_{k+1} \] \[(d_1 + c_{k+1}r_{1k})q_1 + \cdots (d_k+c_{k+1}r_{kk+1})q_k + c_{k+1}r_{k+1, k+1}q_{k+1} = 0
    \]
        Again these vectors, $q_i$, are linearly independent so we must have that $c_{k+1} = 0$ since $r_{k+1, k+1} \neq 0$. Therefore we are left with 
    \[
        d_1q_1 + \cdots d_kq_k = 0
    \]
    We already have that to satisfy this, $d_1 = \cdots = d_k = 0$, thereby we have immediately that the vectors $a_1, \cdots, a_{k+1}$ are linearly independent. Therefore, by inductive argument we have that all $n$ vectors $a_i$ constructed this way from the reduced QR factorization will be linearly independent. As a corallary to this finding, we find that $A$ is rank $n$ by the definition of rank and it being that $A$ is composed of $n$ linearly independent column vectors. 
    \vspace{10pt}

    ($\impliedby$) All diagonal entries of $\hat{R}$ are non-zero if $A$ is rank $n$. 

    Assume by the way of contradiction that both $A$ is rank $n$ and that $\hat{R}$ has at least one diagonal entry, $r_{kk} = 0$. 
    We look at the construction and linear dependence of the column vectors of $A$. Look specifically at $a_k$. We have, 
    \[
        a_k = r_{1k} q_1 + \cdots + r_{k-1, k} q_{k-1} + r_{kk} q_k
    \]
    \[
        c_1a_1 + \cdots + c_ka_k = 0
    \]
    Notice that since $r_{kk} = 0$ $a_k$ is only constructed of $q_1, \cdots, q_{k-1}$. 
    \[
        c_1a_1 + \cdots + c_ka_k = (c_1r_{11} + \cdots + c_kr_{1k})q_1 + \cdots + (c_{k-1}r_{k-1, k-1} + c_kr_{k-1, k})q_{k-1} = 0 
    \]
    We must have again that, $d_1 = \cdots = d_{k-1} = 0$. We then chose $c_k = 1$ for simplicity and obtain a system of equations.
    \[
        c_1r_{11} + \cdots + r_{1k} = \cdots =  c_{k-1}r_{k-1, k-1} + r_{k-1, k} = 0
    \]
    Notice that we have $k-1$ equations with $k-1$ unknowns, so we are guaranteed a solution exists such that at least one $c_i \neq 0$. i.e. 
    \[
        c_{k-1} = -\frac{r_{k-1, k}}{r_{k-1, k-1}} 
    \]
    Therefore we have that the set of column vectors $a_1,\cdots, a_k$ are linearly dependent. Therefore, we have at most that $A$ is rank $n-1$ (Take $\{a_1, \cdots, a_{k-1}, a_{k+1}, \cdots a_n\}$ and check their dependency. They may be linearly independent!). Therefore we have reached a contradiction. If $A$ is full rank (rank = $n$) we cannot have that any diagonal elements of $\hat{R}$ are zero as it would reduce the rank of $A$ by at least one. If $A$ is full rank, $\hat{R}$ must have nonzero diagonal entries. 

\end{proof}

\item Suppose $\hat{R}$ has $k$ nonzero diagonal entries for some $k$ with $0 \le k < n$. What does this imply about the rank of $A$? Exacktly $k$? At least $k$? At most $k$? Give a precise answer and prove it.

    \begin{proof}

    (Case: rank $k$) 

    The goal is to show with two cases that such a matrix can be constructed with rank $n-1$ and one with rank $k$. Therefore stating that the rank of $A$ is at least $k$.  
    Let us look at the case where $\hat{R}$ is a matrix composed of zeros entirely except for $k$ entries along the diagonal.
    \[
        A = \hat{Q}\hat{R}
    \]
    \[
        A = \left[ a_1 \Big | \cdots \Big | a_n \right]
    \]
    Notice that only $k$ column vectors of $A$ are nonzero by this construction of $A$, and $\hat{R}$. Therefore the column vectors which are zero vectors are not linearly independent with each other nor the nonzero column vectors of $A$. So we must have that there are $k$ linearly independent column vectors in $A$. Therefore $A$ is rank $k$. To demonstrate this formally we have
    \[
        c_1a_1 + \cdots c_na_n = \sum_{i, r_{ii} \neq 0} c_ir_{ii}q_i = 0
    \]
    We must have by the linear independence of $q_i$ that $c_ir_{ii} = 0$ for this to be true, but then $c_i = 0$. Since there are $k$ terms in this sum, there are therefore $k$ linearly independence vectors in $A$. 

    (Case: rank $n-1$)

    We next take a case for $\hat{R}$ that will produce $A$ rank $n-1$. We chose an $\hat{R}$, complete with k nonzero entries on the diagonal and zero's above the diagonal for those $k$ columns. For the columns with zero's on the diagonal we demonstrate a particular form for them. For the first column with a zero on the diagonal, the form is not very important. Suppose this is column $i$. Look at the next column with a zero on the diagonal, suppose it is column $j$.  Let column $j$, $r_j$ be of the following form.
    \[
        r_j = \left[\begin{array}{c}
                    0 \\
                    \vdots  \\
                    0 \\
                    r_{i, j} \\
                    0 \\
                    \vdots \\
                    0 
                    \end{array}\right]
    \] 
    These columns are such that if column $c_j$ was in the $i-th$ column rather than the $j-th$ it would resemble an diagonal matrix with nonzero diagonals except for the very last column with one zero on the diagonal (lets denote this column $r_z$). That is, if we permuted the columns of $\hat{R}$ we could obtain a matrix $\hat{R}'$ such that only one column of $\hat{R}'$ has a diagonal entry of zero.  This would produce a matrix $A'$ with the corresponding columns permuted in the same way. Notice however, that $A'$ has the same rank as $A$. That is, it contains the same column vectors, just in a different order. Notice that besides the one column with a zero along the diagonal (lets call this column $a_z$), we have that $\hat{R}'$ is a diagonal matrix. Therefore we have the columns of $A'$ are such that, 
    \[
        a_i' = r_{ii}'q_i, \quad  a_z = r_{1z}'q_1 + \cdots + r_{z-1z}'q_{z-1}
    \]
    We have that $r_{ii}' \neq 0$, so 
    \[
        \sum_{1 \le i \le n, i \neq z} c_ia_i' = \sum_{1 \le i \le n, i \neq z} d_1q_i = 0, \quad (d_i \propto c_i), \quad \text{iff}  \quad c_i = 0, \quad  \forall i
    \]
    Notice that this linear combination (sum) has $n-1$ terms in it, therefore we have that $A'$ is rank $n-1$ and therefore so is $A$. This ultimately implies that the rank of $A$ is bounded on the lower end by $k$ and on the upper end by $n-1$. 
    \end{proof}
\end{enumerate}


\item % start of problem 4
Determine the (i) eigenvalues, (ii) determinant, and (iii) singular values of a Householder reflector. For the eigenvalues, give a geometric argument as well as an algebraic proof. 

\begin{proof}
    (i) Eigenvalues 

    We start with the definition of a householder reflector for a unit vector $x$. Take $H = \I - 2xx^T$ with an eigenvalue-eigenvector pair ($\lambda, v$) such that $Hv = \lambda v$. 
    \[
        Hv = (\I - 2xx^T)v = v - 2xx^Tv = v - 2(x, v)x = \lambda v  
    \]
    \[
        -2(x, v) x = (\lambda - 1) v
    \]
    We again have a case where $x$ and $v$ are vectors connected by scalar arguments. We must have that $x$ and $v$ are colinear. We take the two cases, $x$ and $v$ are colinear, $x$ and $v$ are orthogonal. 
    \[
        v = \alpha x, \quad -2(x, v) = -2\alpha
    \]
    \[
        -2\alpha = (\lambda - 1) \alpha 
    \]
    \[
        \lambda = -1
    \]
    Therefore if $x$ and $v$ are colinear we have that $v$ is an eigenvector of $H$ and that its eigenvalue is $\lambda = -1$. We look at the next case, $x$ and $v$ are orthogonal, therefore $(x, v) = 0$. 
    \[
        -2(0)x = (\lambda - 1)v \implies \lambda - 1 = 0
    \]
    \[
        \lambda = 1 
    \]
    Therefore we have that if $x$ and $v$ are orthogonal that the eigenvalue corresponding to $v$ is equal to 1. 
    \vspace{5pt}

    (ii) Determinant 

    Next we look at the determinant of $H$. We have that from exercise one that $H$ is unitary (orthogonal) and symmetric. Therefore (going in one direction) that $H^{-1} = H^*$. This is because $H^*H = \I = H^{-1}H$. Next we also have that for any matrix A, $\det(A^*) = \overline{\det(A)}$. We also have that, $\det(A)\det(A^{-1}) = 1$.
    \[
        \det(H^{-1}H) = \det(H^{-1})\det(H) = 1
    \]
    \[
        \overline{\det(H)}\det(H) = 1 
    \]
    \[
        \det(H)^2 = 1 \implies \det(H) = \pm 1
    \]
    
    (iii) Singular Values

    We have from the proof in exercise one, we have that $H \in \R^{m\times m}$ is a unitary (orthogonal) matrix. We have therefore that $H$ preserves the length of vectors under transformation. We also look at the singular value decomposition of $H$. 
    \[
        H = U\Sigma V^T, \quad ||Hv|| = ||v||, \forall v \in \R^{m}
    \]
    Let us look at a specific vector $v_i$ now such that $v_i$ is i-th column vector of $V$. 
    \[
        ||Hv|| = ||U\Sigma V^Tv_i|| = ||u_i\sigma_{ii}|| 
    \]
    We recover the scalar-vector product, $u_i\sigma_{ii}$ where $u_i$ is the i-th column vector of $U$ and $\sigma_{ii}$ is the i-th diagonal element of $\Sigma$. We return to the fact that by the Singular Value Decomposition Theorem, that $U, V$ are unitary, that is they are composed of orthogonal column vectors with norm of 1. Therefore we have, 
    \[
        ||Hv|| = ||v_i|| = ||\sigma_{ii}u_i|| = |\sigma_{ii}|||u_i||
    \]
    \[
        1 = |\sigma_{ii}|1, \implies \sigma_{ii} = \pm 1
    \]
    Therefore since our choice of $v$ was arbitrary among the column vectors of $V$ we have that this example exhausts all singular values for $H$. Thus the singular values of $H$ are $\pm 1$. It can even be argued that the plus minus in this context does not matter. Since singular values are scalars which in a transformation from one vector basis to another scale the vector in the resulting basis. The vectors in the output basis are orthogonal so scaling one vector say by $-1$ would not make that basis linearly dependent. Therefore we claim that any $u_i$ will absorb the sign of $\sigma_{ii}$ (Also because of the fact that singular values are always positive). So it is as simpler to claim that, 
    \[
        \sigma_{ii} = 1.
    \]
    Therefore the singular values of $H$ are such that, $\sigma_{ii} = \sigma_i = 1$. 
\end{proof}

\item  % start of problem 5
Let $A \in \R^{m\times n}$. Show that $\cond(A^TA) = \left(\cond(A)\right)^2$. 

    \begin{proof}
        We start with the singular value decomposition of $A$. 
        \[
            A = U\Sigma V^T, \quad U, V \text{ unitary}
        \]
        \[
            A^TA = V\Sigma U^T U \Sigma V^T = V\Sigma^2V^T
        \]
        Notice that this is a singular value decomposition for $A^TA$ since $V, V^T$ are unitary matrices and $\Sigma^2$ is a diagonal matrix with positive or zero entries along the diagonal. We look at the fact that the condition number of a matrix $A$ is  proportional to the two norms of $A$ and $A^{-1}$. 
        \[
            \cond(A^TA) = ||A^TA||_2 \cdot ||(A^TA)^{-1}||_2
        \]
        We will also use the fact that the two-norm of a matrix is equal to its largest singular value. We now look for $(A^TA)^{-1}$. 
        \[
            (A^TA)^{-1} (A^TA) = \I
        \]
        \[
            U_1U_2U_3 V \Sigma^2V^T = \I
        \]
        Very evidently from this assumption we can pick three matrices to invert $A^TA$. We take $U_3 = V^T$, $U_2 = \Sigma^{-2}$ (this inverse exists because $\Sigma$ is diagonal), $U_1 = V$ (assuming that $V$ is invertible). Thus we have, 
        \[
            (A^TA)^{-1} = V\Sigma^{-2}V^T
        \]
        Notice that this is also a singular value decomposition for $(A^TA)^{-1}$ since both $V, V^T$ are unitary and $\Sigma^{-2}$ is still diagonal. Notice however the largest singular values for $A^TA, (A^TA)^{-1}$ are $\sigma_1^2, \frac{1}{\sigma_k^2}$ respectively. Therefore we go back to the condition number. 
        \[
            \cond(A^TA) = ||A^TA||_2 \cdot ||(A^TA)^{-1}||_2 = \sigma_1^2 \frac{1}{\sigma_k^2} = \left(\frac{\sigma_1}{\sigma_k}\right)^2 = \left(\cond(A)\right)^2
        \]  
        This last bit ($\cond(A) = \frac{\sigma_1}{\sigma_k}$) is taken from a proof in lecture (I don't know where but its fairly evident using a singular value decomposition in almost exactly the same way as we are presenting this argument). 
   \end{proof}

\end{enumerate}

\end{document}
