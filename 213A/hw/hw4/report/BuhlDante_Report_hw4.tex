\documentclass{article}
\usepackage{graphicx} % Required for inserting images
\usepackage[margin=1in]{geometry}
\usepackage{amsmath}
\usepackage{amsthm}
\usepackage{amssymb}
\usepackage{amsfonts}
\usepackage{verbatim}
\usepackage{xcolor}

\title{Homework 4: Report}
\author{Dante Buhl}
\date{Feb. $26^{th}$ 2024}


\DeclareMathOperator{\cond}{cond}

\begin{document}

\newcommand{\bs}[1]{\boldsymbol{#1}}
\newcommand{\bmp}[1]{\begin{minipage}{#1\textwidth}}
\newcommand{\emp}{\end{minipage}}
\newcommand{\R}{\mathbb{R}}
\newcommand{\C}{\mathbb{C}}
\newcommand{\N}{\mathcal{N}}
\newcommand{\I}{\mathrm{I}}
\newcommand{\K}{\bs{\mathrm{K}}}
\newcommand{\m}{\bs{\mu}_*}
\newcommand{\s}{\bs{\Sigma}_*}
\newcommand{\dt}{\Delta t}
\newcommand{\tr}[1]{\text{Tr}(#1)}
\newcommand{\Tr}[1]{\text{Tr}(#1)}

\maketitle


\setcounter{section}{1}

\section{Cholesky Solution of the least-squares problem}
\begin{enumerate}
\item 

\end{enumerate}

\section{QR Solution of the least-squares problem}
\begin{enumerate}
\item 

\end{enumerate}

\section{Theory Problems}
\begin{enumerate}

\item %start of problem 1
Show that is $P$ is an orthogonal projector, then $\I - 2P$ is unitary. 

\begin{proof}
    We begin with the definition of a unitary matrix. We have a unitary matrix $Q$ is a matrix such that $Q^*Q = \I$. We now look for this quality in $(\I - 2P)$. 
    \[
        (\I - 2P)^*(\I - 2P) = (\I - 2P^T)(\I - 2P) = (\I - 2P)(\I - 2P)
    \]
    The above quality $P^T = P$ is from the fact that $P$ is an orthogonal projector. We also have the quality that $P^2 = P$.
    \[
        (\I - 2P)(\I - 2P) = \I(\I - 2P) - 2P(\I - 2P) = \I - 2P - 2P + 4P^2
    \]
    \[
        = \I - 4P +4P = \I
    \]  
    We have recovered the condition for unitary matrices. We can therefore declare $(\I - 2P)$ a unitary matrix. 
\end{proof}


\item % start of problem 2
Let $P \in \R^{m \times m}$ be a nonzero projector. 
\begin{enumerate}
    \item Show that $||P||_2 \ge 1$, with equality if and only if $P$ is an orthogonal projector. 

    \item Show that if $P$ is an orthogonal projector, then P is semi-positive deinite with its eigenvalues either zero or 1. 
        \begin{proof}
            We look at the vector product definition of $P$. $P = xx^T$ for a unit vector $x$. 
            \[
                (v, Pv) = v^Txx^Tv = (v, x)(x, v) = (x, v)^2 \ge 0
            \]
            Since our choice of $v$ was arbitary we have that $P$ is semi-positive definite. 
            
            Next we look at the eigenvalues of $P$. Say that we have an arbitary eigenvalue-eigenvector pair ($\lambda, v$) for $P$ such that $v \neq \vec{0}$ (obviously). We have,
            \[
                Pv = \lambda v
            \]
            \[
                Pv = xx^Tv = (x, v)x = \lambda v
            \]
            Notice that $(x, v)$ and $\lambda$ are scalars. This implies that $x$ and $v$ are colinear but this was not an assumption made. Therefore we are left with two cases: $v$ and $x$ are colinear, or $v$ and $x$ are orthogonal. Let's look at the first case, $v = \alpha x$.
            \[
                (x, v)x = \alpha x = \alpha\lambda x
            \] 
            \[
                x = \lambda x \implies \lambda = 1
            \]
            We find that for all vectors colinear to $x$ are eigenvectors with eigenvalue 1. We look at the other case. If $x$ and $v$ are orthogonal we have $(x, v) = 0$. 
            \[
                \vec{0} = \lambda v \implies \lambda = 0
            \]
            Therefore all vectors orthogonal to $x$ will be eigenvectors with $\lambda = 0$. 
        \end{proof}
\end{enumerate}

\item   % start of problem 3
Let $A \in \R^{m \times m}$ with $m \ge n$, and let $A = \hat{Q}\hat{R}$ be a reduced $QR$ factorization. 

\begin{enumerate}
\item Show that $A$ has rank $n$ if and only if all the diagonal entries of $\hat{R}$ are nonzero.

\item Suppose $\hat{R}$ has $k$ nonzero diagonal entries for some $k$ with $0 \le k < n$. What does this imply about the rank of $A$? Exacktly $k$? At least $k$? At most $k$? Give a precise answer and prove it.
\end{enumerate}


\item % start of problem 4
Determine the (i) eigenvalues, (ii) determinant, and (iii) singular values of a Householder reflector. For the eigenvalues, give a geometric argument as well as an algebraic proof. 

\begin{proof}
    (i) Eigenvalues 

    We start with the definition of a householder reflector for a unit vector $x$. Take $H = \I - 2xx^T$ with an eigenvalue-eigenvector pair ($\lambda, v$) such that $Hv = \lambda v$. 
    \[
        Hv = (\I - 2xx^T)v = v - 2xx^Tv = v - 2(x, v)x = \lambda v  
    \]
    \[
        -2(x, v) x = (\lambda - 1) v
    \]
    We again have a case where $x$ and $v$ are vectors connected by scalar arguments. We must have that $x$ and $v$ are colinear. We take the two cases, $x$ and $v$ are colinear, $x$ and $v$ are orthogonal. 
    \[
        v = \alpha x, \quad -2(x, v) = -2\alpha
    \]
    \[
        -2\alpha = (\lambda - 1) \alpha 
    \]
    \[
        \lambda = -1
    \]
    Therefore if $x$ and $v$ are colinear we have that $v$ is an eigenvector of $H$ and that its eigenvalue is $\lambda = -1$. We look at the next case, $x$ and $v$ are orthogonal, therefore $(x, v) = 0$. 
    \[
        -2(0)x = (\lambda - 1)v \implies \lambda - 1 = 0
    \]
    \[
        \lambda = 1 
    \]
    Therefore we have that if $x$ and $v$ are orthogonal that the eigenvalue corresponding to $v$ is equal to 1. 
    \vspace{5pt}

    (ii) Determinant 

    Next we look at the determinant of $H$. We have that from exercise one that $H$ is unitary (orthogonal) and symmetric. Therefore (going in one direction) that $H^{-1} = H^*$. This is because $H^*H = \I = H^{-1}H$. Next we also have that for any matrix A, $\det(A^*) = \overline{\det(A)}$. We also have that, $\det(A)\det(A^{-1}) = 1$.
    \[
        \det(H^{-1}H) = \det(H^{-1})\det(H) = 1
    \]
    \[
        \overline{\det(H)}\det(H) = 1 
    \]
    \[
        \det(H)^2 = 1 \implies \det(H) = \pm 1
    \]
    
    (iii) Singular Values

    We have from the proof in exercise one, we have that $H \in \R^{m\times m}$ is a unitary (orthogonal) matrix. We have therefore that $H$ preserves the length of vectors under transformation. We also look at the singular value decomposition of $H$. 
    \[
        H = U\Sigma V^T, \quad ||Hv|| = ||v||, \forall v \in \R^{m}
    \]
    Let us look at a specific vector $v_i$ now such that $v_i$ is i-th column vector of $V$. 
    \[
        ||Hv|| = ||U\Sigma V^Tv_i|| = ||u_i\sigma_{ii}|| 
    \]
    We recover the scalar-vector product, $u_i\sigma_{ii}$ where $u_i$ is the i-th column vector of $U$ and $\sigma_{ii}$ is the i-th diagonal element of $\Sigma$. We return to the fact that by the Singular Value Decomposition Theorem, that $U, V$ are unitary, that is they are composed of orthogonal column vectors with norm of 1. Therefore we have, 
    \[
        ||Hv|| = ||v_i|| = ||\sigma_{ii}u_i|| = |\sigma_{ii}|||u_i||
    \]
    \[
        1 = |\sigma_{ii}|1, \implies \sigma_{ii} = \pm 1
    \]
    Therefore since our choice of $v$ was arbitrary among the column vectors of $V$ we have that this example exhausts all singular values for $H$. Thus the singular values of $H$ are $\pm 1$. It can even be argued that the plus minus in this context does not matter. Since singular values are scalars which in a transformation from one vector basis to another scale the vector in the resulting basis. The vectors in the output basis are orthogonal so scaling one vector say by $-1$ would not make that basis linearly dependent. Therefore we claim that any $u_i$ will absorb the sign of $\sigma_{ii}$ (Also because of the fact that singular values are always positive). So it is as simpler to claim that, 
    \[
        \sigma_{ii} = 1.
    \]
    Therefore the singular values of $H$ are such that, $\sigma_{ii} = \sigma_i = 1$. 
\end{proof}

\item  % start of problem 5
Let $A \in \R^{m\times n}$. Show that $\cond(A^TA) = \left(\cond(A)\right)^2$. 

We start with 

\end{enumerate}

\end{document}
