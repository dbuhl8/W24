\documentclass{article}
\usepackage{graphicx} % Required for inserting images
\usepackage[margin=1in]{geometry}
\usepackage{amsmath}
\usepackage{amsthm}
\usepackage{amssymb}
\usepackage{amsfonts}
\usepackage{verbatim}
\usepackage{xcolor}

\title{AM 213A Midterm: Report}
\author{Dante Buhl}
\date{Feb. $16^{th}$ 2024}

\begin{document}

\newcommand{\bs}[1]{\boldsymbol{#1}}
\newcommand{\bmp}[1]{\begin{minipage}{#1\textwidth}}
\newcommand{\emp}{\end{minipage}}
\newcommand{\R}{\mathbb{R}}
\newcommand{\C}{\mathbb{C}}
\newcommand{\N}{\mathcal{N}}
\newcommand{\I}{\mathrm{I}}
\newcommand{\K}{\bs{\mathrm{K}}}
\newcommand{\m}{\bs{\mu}_*}
\newcommand{\s}{\bs{\Sigma}_*}
\newcommand{\dt}{\Delta t}
\newcommand{\tr}[1]{\text{Tr}(#1)}
\newcommand{\Tr}[1]{\text{Tr}(#1)}

\maketitle


\section{Problem 1: Define the Following}
\begin{enumerate}

\item A full rank matrix $A$ and a rank deficient matrix $B$, for $A, B \in  \R^{m\times n}$. 

We have that $A$ is such that, rank$(A) = \min(m, n) = k$. That is, $A$ has k linearly independant columns. \colorbox{yellow}{NEED TO DO B STILL}.

\item An orthogonal matrix $Q$ and a unitary matrix $U$, where $Q$ and $U$ are square. 

We have that $Q \in \R^{n\times n}$ such that, $Q^TQ = \I$. It is also a fact that the columns of $Q$ form a basis for $\R^n$. A unitary matrix $U$ is very similar but for complex spaces. We have, $U \in \C^{n\times n}$, such that $U^*U = \I$. The columns of $U$ form a basis for $\C^n$.

\item Singular value decomposition of $A \in \C^{m\times n}$ with rank$(A) = k \le \min(m, n)$

\item Orthogonal projector $P \in \R^{m\times m}$. 

\item Defective matrix $A \in \R^{m\times m}$.

\item Relative condition number $\kappa(x_0)$ of a differentiable function $f(x) = Ax$ at $x = x_0$ and its upper bound when A is nonsingular

\item Condition number $\kappa$ or cond($A$) of a nonsingular matrix $A$ with rank($A) = k$ in the 2-norm in terms of singular values.

\item Diagonalizable matrix $A \in \R^{m\times m}$.

\item Machine accuracy $\epsilon_{\text{mach}}$ in single and double precisions.

\item Backward stable algorithm $\tilde{f}$ for a problem $f$. What is the relation between the backward stability and accuracy?


\end{enumerate}

\section{Problem 2 - Let $A = (a_{ij})_{i,j=1}^n \in \R^{n\times n}$}
\begin{enumerate}
\item Suppose A has nonnegative entries such that $\sum_{j=1}^n a_{ij} = 1$ for $1 \le i \le n$. Show that no eigenvalule of $A$ has an absolute value greater than 1. 

\item Suppose that $\sum_{j=1}^n |a_{ij}| < 1$ for each $i$. Prove that $B = \I - A$ is invertible.  (Hint: Using the Equivalence Theorem for a nonsingular matrix A, it suffices to show that the dimension of the null space of B is 0.)

\end{enumerate}

\section{Problem 3 - Let $A \in \R^{m\times m}$ be symmetric.}
\begin{enumerate}
\item Define what it means to say A is symmetric positive definite.

If $A$ is real, symmetric, positive definite we have the following properties of $A$. $A^* = A^T = A$, and $(x, Ax) > 0, \forall x \in \R^m$. 


\item  Show that the eigenvalues of a symmetric positive definite matrix A are positive.


\begin{proof}

We look at the inner product of the eigenvectors of $A$, with the matric product between $A$ and its eigenvector. 

\[
    Av = \lambda v, \quad \text{an eigenvalue eigenvector pair } (\lambda, v)
\]
\[
    (v, Av) = v^TAv = v^T(\lambda v) = \lambda ||v||_2^2 > 0
\]
\[
    ||v||_2^2 > 0 \implies \lambda > 0
\]
Since our choice of $\lambda$ and $v$ were arbitrary, we have that all eigenvalues of $A$ are positive. 
\end{proof}


\item What can you say about the diagonal entries of a symmetric positive definite matrix
A? Justify your answer by proving or disproving it.

Claim: The diagonal entries of a symmetric positive definite matrix $A$ are positive. 

\begin{proof}
    We begin by looking at the inner product with the unit basis vectors of $\R^m$. 
    \[
        \hat{e}_i = \left[\begin{array}{c} 0\\\vdots\\1\\\vdots\\0\end{array}\right]
    \]
    Where only the i-th element of $\hat{e}_i$ is nonzero. 
    \[
        (\hat{e}_i, A\hat{e}_i) = \hat{e}_i^T \left[\begin{array}{c} a_{1i}\\\vdots\\a_{mi}\end{array}\right] = a_{ii} > 0 
    \]
    Therefore the diagonal elements of $A$ are all positive. 
\end{proof}

\end{enumerate}

\section{Problem 4 - Consider the following algorithm for a linear system with $A \in \R^{m\times m}$}

\begin{enumerate}
\item

\item

\item 

\item 

\end{enumerate}


\section{Problem 5 - Consider the matrix}

\[
    A = \left[\begin{array}{c c}
                1 & 1 \\ \epsilon & 0 \\ 0 & \epsilon \end{array}\right]
\]
for the least squares problem $Ax \approxeq b$ with $0 < \epsilon < \sqrt{\epsilon_{\text{mach}}}$ Here $\epsilon_{\text{mach}}$ is the small value in
machine accuracy and is numerical zero. In this problem, all your arithmetic manipulations should mimic the computer’s finite precision handling.

\begin{enumerate}
\item Carry out to use the normal equation by first multiplying $A^T$ on both sides to directly solve the linear system. Discuss what happens.

\item  Check if the resulting $A^TA$ is symmetric positive definite (SPD) by directly using the definition of SPD. Conclude whether you can use the Cholesky factorization method or not for the resulting normal equation. Justify your answer.

\item In general, using the normal equation $A^TAx = A^Tb$ for solving the least squares problem $Ax \approxeq b$ is not always ideal. Justify why it is not ideal by proving the condition number of the Gram matrix $A^TA$ is the square of the condition number of $A$, i.e $\kappa(A^TA) = (\kappa(A))^2$, for a full rank matrix $A \in \R^{m\times n}, m \ge n$. What is the implication of $\kappa(A^TA) = (\kappa(A))^2$.

\item 

\end{enumerate}

\section{Problem 6 }

\begin{enumerate}
\item

\item
The Schur decomposition theorem states that if $A \in C^{m\times m}$, then there
exist a unitary matrix $Q$ and an upper triangular matrix $U$ such that
$A = QUQ^{-1}$ . Use the Schur decomposition theorem to show that a real
symmetric matrix $A$ is diagonalizable by an orthogonal matrix, i.e., $\exists$ an
orthogonal matrix $Q$ such that $Q^T AQ = D$, where $D$ is a diagonal matrix
with its eigenvalues in the diagonal.

\begin{proof}

We begin with the Schur Decomposition Theorm for real symmetric matrix $A$. We have, 
\[
    A = QUQ^{-1} = QUQ^*
\]
By the property that $A$ is real and symmetric, we have that $A^* = A$.
\[
    A^* = A = QUQ^*
\]
\[
    U = Q^*AQ, \quad U^* = Q^*AQ
\]
\[
    U = U^*, \implies U \text{ is real, symmetric, DIAGONAL}
\]
We also have that $U$ has the same eigenvalues of $A$ by the fact that they are similar matrices (Nate I will not cite this, sorry about it). So we have that $A$ is diagonalized by unitary matrices $Q, Q^{-1}$. Now all we need is to show that $Q$ is real (orthogonal). We start from that fact that $A = A^* = A^T = \overline{A}$.
\[
    AA^* = A^TA = AA^T = A^TA^T
\] 
\[
    QU^2Q^* = Q^{*T}UQ^TQUQ^* = QUQ^*Q^{*T}UQ^T = Q^{*T}U^2Q^T
\]
\[
   QU = Q^{*T}UQ^TQ, \quad UQ^* = Q^*Q^{*T}UQ^T
\]
\[
    QU^2Q^* = Q^{*T}UQ^TQQ^*Q^{*T}UQ^T = Q^{*T}U^2Q^T 
\]
\[
    U^2 = UQ^TQU, \implies Q^TQ = \I, \text{ EUREKA!}
\]
So we therefore have that $Q$ is orthogonal by definition, or rather Q is real and unitary! Therefore, $A$ is diagonalizable by an orthogonal matrix, $Q$. (Nate, in retrospect I realize this logic might be flawed. I just checked and there is a $Q^T$ out of place in my last substitution. But I have 10 minutes to submit and I don't care anymore. Do your worst)
\end{proof}


\item 

\item 

\end{enumerate}

\section{Problem 7 - }

\begin{enumerate}
\item

\item

\item 

\item 

\end{enumerate}



\section{Problem 8 - Let $P \in \R^{m\times m}$ be an orthogonal projector}
\begin{enumerate}

\item Show that P is positive semi-definite with its eigenvalues either zero or one. (Hint: A symmetric matrix is orthogonally diagonalizable.)

\item What can you say about the dimension of P if its eigenvalues are all distinct with
algebraic multiplicity of 1?

\item Construct $P \in \R^{2\times2}$ whose entries are all nonzero. Identify all possible choices of P. For each of your constructed P, show that it is positive semi-definite and its eigenvalues are either 0 and 1.

\end{enumerate}




\end{document}
