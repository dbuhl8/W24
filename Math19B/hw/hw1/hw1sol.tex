\documentclass[11pt]{article}

%\usepackage{xcolor} 
\usepackage{amssymb,amsfonts,amsmath,mathrsfs,mathtools}
\usepackage{graphicx, epsfig, epstopdf}
\usepackage{bm}
\usepackage[margin=1in,papersize={8.5in,11in}]{geometry}
\usepackage{times}
\usepackage{xcolor}
\usepackage{color}
\usepackage[pdftex, plainpages=false, colorlinks=true, linkcolor=blue, citecolor=blue, bookmarks=false]{hyperref}
\renewcommand{\thefootnote}{\fnsymbol{footnote}}

\newcommand{\vs}{\vspace{0.2cm}}
\newcommand{\noter}[1]{\textcolor{red}{{#1}}}
\newcommand{\noteb}[1]{\textcolor{blue}{{#1}}}
\newcommand{\bmp}[1]{\begin{minipage}{#1\textwidth}}
\newcommand{\emp}{\end{minipage}}
\newcommand{\hleq}[1]{\colorbox{yellow}{\[#1\]}}

\makeatother

\def\hs{\hspace{1cm}}

\renewcommand{\baselinestretch}{1.4}

\title{Homework 1}
\author{Gomez - Math 19B}
\date{Due: Jan 19th, 2024}

\begin{document}

\maketitle

\noindent 
\normalsize
All exercises are taken from Section 5.1 in the textbook

\begin{enumerate}
\item 
Exercise 8:
Let $f(x) = x^2 + x - 2$.
\begin{itemize}
\item Calculate $R_3$ and $L_3$ over [2, 5]. NOTE: $R_3$ and $L_3$ denote the right and left endpoint approximations of the Area under $f(x)$, each made of three rectangles/intervals. 

\bmp{.45}
\begin{align*}
    R_3 &= \Delta x (h_1 + h_2 + h_3) \\
    R_3 &= (1)(f(3) + f(4) + f(5)) \\
    R_3 &= 10 + 18 + 28 = 56
\end{align*}
\emp
\bmp{.45}
\begin{align*}
    L_3 &= \Delta x (h_1 + h_2 + h_3) \\ 
    L_3 &= (1)(f(2) + f(3) + f(4)) \\ 
    L_3 &= 4 + 10 + 18 = 32
\end{align*}
\emp

\colorbox{yellow}{$R_3 = 56$, $L_3 = 32$}
\item Sketch the graph of f and the rectangles that make up each approximation. Is the area under the graph larger or smaller than $R_3$? \colorbox{yellow}{The area under f(x) is smaller than $R_3$}

Than $L_3$ ? \colorbox{yellow}{The area under f(x) is larger than $L_3$}
\end{itemize}

\item 
Derive the Midpoint Approximation Formula, $M_n$, for a function, $f(x)$, on the interval [$a, b$] through the process of this problem. 
\begin{itemize}
    \item Find the base width for each rectangle, $\Delta x$, in terms of $n$.

    \[
    \Delta x = \frac{\text{length of interval}}{\text{number of rectangles}} = \frac{b - a}{n}
    \]
    \colorbox{yellow}{$\Delta x = \frac{b-a}{n}$}

    \item Find the height of the i-th rectangle, $h_i$, in terms of $f$ and $x_i$, where $x_i$ denotes the midpoint of each interval.

    \[
        h_i = \text{height of function at midpoint} = f(x_i)
    \]
    \colorbox{yellow}{$h_i = f(x_i)$}

    \item Find the midpoint of the i-th subinterval, $x_i$, in terms of $i, \Delta x$. 

    \[
    x_i = \text{midpoint of i-th rectangle} = a + \Delta x \cdot \frac{i}{2}
    \]

    \colorbox{yellow}{$x_i = a + \Delta x \cdot \frac{i}{2}$}
    \item Find the area of the i-th rectangle, $A_i$, in terms of $i, f,$ and $\Delta x$ using the equation for the area of a rectangle.

    \[
    A_i = b \cdot h_i = \Delta x \cdot f(x_i) = \Delta x \cdot f(a + \Delta x \cdot \frac{i}{2})
    \]

    \colorbox{yellow}{$A_i = \Delta x \cdot f(a + \Delta x \cdot \frac{i}{2})$}

    \item Express the midpoint approximation, $M_n$, as the sum of the areas of all rectangles in the partition in terms of $i, \Delta x$, and $f$.

    \[
    M_n = \sum_{i = 1}^{n} \Delta x \cdot f(a + \Delta x \cdot \frac{i}{2}) = \Delta x \sum_{i=1}^n f(a + \Delta x \cdot \frac{i}{2})
    \]

    \colorbox{yellow}{$M_n = \Delta x \sum_{i = 1}^n f(a + \Delta x \cdot \frac{i}{2}) = \frac{b-a}{n} \sum_{i=1}^n f(a + \frac{i(b-a)}{2n})$}
    Either answer is fine, since I forgot to specify to write the answer in terms of n, f, and i. 
\end{itemize}


\setcounter{equation}{2}
\textbf{Formulas (3)-(5)}
\begin{equation}
    \sum_{i=1}^Ni = 1 + 2 + \cdots + N = \frac{N(N+1)}{2} = \frac{N^2}{2} + \frac{N}{2}
\end{equation}
\begin{equation}
    \sum_{i=1}^Ni^2 = 1 + 4 + \cdots + N^2 = \frac{N(N+1)(2N+1)}{6} = \frac{N^3}{3} + \frac{N^2}{2} + \frac{N}{6}
\end{equation}
\begin{equation}
    \sum_{i=1}^Ni^3 = 1 + 8 + \cdots + N^3 = \frac{N^2(N+1)^2}{4} = \frac{N^4}{4} + \frac{N^3}{2} + \frac{N^2}{4}
\end{equation}

\item 
Exercise 44: Compute the sum using linearity and the formulas (3)-(5) for power sums covered in the textbook (pg. 304)

\[
    \sum_{i=1}^{10} (i^3 - 2i^2)
\]
\[\sum_{i=1}^{10} (i^3 - 2i^2) = \sum_{i=1}^{10} i^3 - 2\sum_{i=1}^{10} i^2\]

\[=  \frac{10^4}{4} + \frac{10^3}{2} + \frac{10^2}{4} - 2\left[\frac{10^3}{3} + \frac{10^2}{2} + \frac{10}{6}\right]\]

\[= 3025 - 770 = 2255\]

\colorbox{yellow}{$\sum_{i=1}^{10} (i^3 - 2i^2) = 2255$}
\item
Exercise 52: use linearity and formulas (3)–(5) to evaluate the limit.

\[
    \lim_{N\to\infty} \sum_{i=1}^N \frac{i}{N^2}
\]

\[\lim_{N\to\infty} \sum_{i=1}^N \frac{i}{N^2} = \lim_{N\to\infty} \frac{1}{N^2} \sum_{i=1}^N i\]

\[= \lim_{N \to \infty} \frac{1}{N^2} (\frac{N^3}{3} + \frac{N^2}{2} + \frac{N}{6})\]

\[= \lim_{N \to \infty} \frac{N}{3} + \frac{1}{2} + \frac{1}{6N}  = \infty + \frac{1}{2} + 0 = \text{DNE}\]

\colorbox{yellow}{$\lim_{N\to\infty} \sum_{i=1}^N \frac{i}{N^2} : \text{DNE}$}





\end{enumerate}
%

%
\end{document}
%
%
