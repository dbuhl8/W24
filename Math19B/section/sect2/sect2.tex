\documentclass{beamer}

\title{Section 2}
\author{TA: Dante Buhl}
\institute{UCSC Math-19B}
%\date{Week 2}
\graphicspath{ {./images/} }
%\usepackage{xcolor}
\usepackage{verbatim}
\usetheme{Goettingen}
\usecolortheme{seahorse}
\usefonttheme{serif}
%\usepackage[x11names]{xcolor}
\usepackage{amsmath}
\usepackage{pifont}


\begin{document}

\newcommand{\bmp}[1]{\begin{minipage}{#1\textwidth}}
\newcommand{\emp}{\end{minipage}}


%Title
\frame{\titlepage}

\section{Welcome to Section!}
\begin{frame}{Plan for Today}
    Topics to Cover
    \begin{itemize}
        \item Review Activity
        \item Substitution Method
    \end{itemize}
    Section Activity 2
    \begin{itemize}
        \item 3 questions
    \end{itemize}
    Assignments
    \begin{itemize}
        \item Homework 2 (Due Fri, Jan. $26^{th}$)
    \end{itemize}
\end{frame}



\begin{frame}{Learning Outcomes}
    \begin{itemize}
        \item Using FTC P.1 and P.2 to compute integrals.
        \item Develop strategies for finding anti-derivatives.
        \item Understand how to use and apply u-substitution for integrals/anti-derivatives.
    \end{itemize}
\end{frame}

\section{Practice}
\begin{frame}{Review Activity}

    \includegraphics[width=.6\textwidth]{activity.png}


\end{frame}

\begin{frame}{U-Substitution}
\scriptsize{
U-Substitution is an extremely useful method for integration. It relies on using a change of variables (substitution) in order to integrate a simpler function. It is essentially the chain rule for derivatives in reverse. }

\normalsize
\vspace{10pt}


The Chain Rule. 

\[
    \frac{d}{dx} f(g(x)) = f'(g(x)) \cdot g'(x)
\]


\end{frame}

\begin{frame}{U-Substitution}

\[
    \int f'(g(x))g'(x) dx
\]
\[
    u = g(x) \to \frac{du}{dx} = g'(x)
\]
\[
    du = g'(x)dx
\]
\[
    \int f'(g(x))g'(x) dx = \int f'(u)du
\]

Example: 

\[
\int 2x\cos{x^2}dx
\]
\[
    u = x^2, \to du = 2xdx
\]
\[
    \int 2x\cos{x^2}dx = \int \cos{u} du = \sin{u} + c = \sin{x^2} + c
\]

\end{frame}

\begin{frame}{Section Activity 2 (Access Code: )}
    
\end{frame}

\end{document}
