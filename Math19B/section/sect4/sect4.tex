\documentclass{beamer}

\title{Section 4}
\author{TA: Dante Buhl}
\institute{UCSC Math-19B}
%\date{Week 2}
\graphicspath{ {./images/} }
%\usepackage{xcolor}
\usepackage{verbatim}
\usetheme{Antibes}
\usecolortheme{beetle}
\usefonttheme{serif}
%\usepackage[x11names]{xcolor}
\usepackage{amsmath}
\usepackage{pifont}


\begin{document}

\newcommand{\bmp}[1]{\begin{minipage}{#1\textwidth}}
\newcommand{\emp}{\end{minipage}}


%Title
\frame{\titlepage}

\section{Agenda}
\begin{frame}{Plan for Today}
    Topics to Cover
    \begin{itemize}
        \item Figuring out working groups
        \item Work and Energy
        \item Numerical Approximations
    \end{itemize}
    Section Activity 4
    \begin{itemize}
        \item 4 questions
    \end{itemize}
    Upcoming Assignments
    \begin{itemize}
        \item Homework 4 (Due Fri, Feb. $9^{th}$)
        \item Project 1 (Due Tues, Feb $20^{th}$).
    \end{itemize}
\end{frame}


\begin{frame}{Scan this to submit working groups form!}

    GET INTO WORKING GROUPS ON CANVAS, ignore that llink below


    https://forms.gle/tGbYvzhE8Eyqxw9c6

\end{frame}

\section{Review}
\begin{frame}{Learning Outcomes}
    \begin{itemize}
        \item Understanding physically motivated integration problems
        \item Differentiating and applying Numerical Integration Techniques
    \end{itemize}
\end{frame}

\begin{frame}{Work and Energy Integrals}

Physics is a language spoken in calculus often consisting of derivatives, integrals, and differential equations. 

Recall that we can relate acceleration $a(t)$, velocity $v(t)$, and position $s(t)$ through derivatives. 
\[
    a(t) = \frac{d v(t)}{dt}, \quad v(t) = \frac{d s(t)}{dt}
\]
Similarly, we can express Work and Energy in terms of integrals. We have, 
\[
    \text{Work, } W = \int_{\mathcal{D}} F dx
\]
Where $\mathcal{D}$ represents the domain over which you integrate (e.g. $x \in \mathcal{D} := [a, b]$) and $F(x)$ represents a force that varies in $x$. 
\end{frame}
\begin{frame}{Work and Energy}
    We can look at this relationship a little more closely. By the fundamental theorem of calculus we should recover force $F(x)$ by taking the derivative of Work. 
    \[
        \frac{d}{dx} W = \frac{d}{dx} \int_a^x F(s)ds = F(x) 
    \]
    Another thing to notice is the units of force and work. In physics, forces are often measures in Newtons ($N$). By integrating in x (i.e. meters, $m$), we obtain a new scientific unit, Joules ($J$). 
    \[
        N = \frac{kg \cdot m }{s^2}
    \]  
    \[
        J = N \cdot m = \frac{kg \cdot m^2}{s^2}
    \]
\end{frame}

\begin{frame}{Numerical Methods for Integrals}

    There are a couple new methods for approximating integrals and quantifying their errors shown in section 7.1 in the book.
    \begin{itemize}
        \item Trapezoidal Rule $T_N$
        \item Simpson's Rule $S_N$
        \item Error Bound, Error($A_N$)
    \end{itemize}
\end{frame}

\begin{frame}{Trapezoidal Rule}
    \[
        A_{Trap} = \frac{1}{2}(w)(h_1 + h_2) = \frac{1}{2}\Delta x (f(x_{i-1}) + f(x_i)) = A_i
    \]
    \[
        \Delta x = \frac{b - a}{N}, \quad x_i = a + i \Delta x 
    \]
    \[
        T_N = \sum_{i=1}^N A_i = \frac{\Delta x}{2}\sum_{i=1}^N (f(x_{i-1}) + f(x_i))
    \]  
    Error Bound for $T_N$
    \[
        \text{Error}(T_N) \le \frac{K_2(b-a)^3}{12N^3}
    \]
    Where $K_2$ is defined to be equal to be the maximum of $|f''(x)|$ on $[a, b]$. 
\end{frame}

\begin{frame}{Simpson's Rule $S_N$}
    Simpson's Rule is obtained from a combination of the Trapezoidal Rule and the Midpoint Approximation. 

    \[
        S_N = \frac{1}{3}T_{N/2} + \frac{2}{3}M_{N/2}, \quad N \text{ even}
    \]
    \[
        \Delta x = \frac{b-a}{N}, \quad  y_i = f(a + i\Delta x)
    \]
    \[
        S_N = \frac{1}{3}\Delta x [y_0 + 4y_1 + 2y_2  + \cdots + 2y_{N-2} + 4y_{N-1} + y_N]
    \]
    
\end{frame}

\begin{frame}{Error Bound}

    Error Bounds are also defined, with $K_2$ being the maximum value of the second derivative $f''(x)$ on $[a, b]$ and $K_4$ being defined the same but with the fourth derivative. 

    \[
        \text{Error}(T_N) \le \frac{K_2(b-a)^3}{12N^2}
    \]
    \[
        \text{Error}(M_N) \le \frac{K_2(b-a)^3}{24N^2}
    \]
    \[
        \text{Error}(S_N) \le \frac{K_4(b-a)^5}{180N^4}
    \]
    \[
        K_2 \ge \max_{x \in [a, b]}\{f''(x)\}, \quad K_4 \ge \max_{x \in [a, b]}\{f^{(4)}(x)\}
    \]


\end{frame}


\section{Attenance}
\begin{frame}{Discussion Section Activity 4}
    \raggedright
    Woah look, the TA is about to write the code on the board!
\end{frame}

\begin{frame}

Better

\end{frame}


\end{document}
