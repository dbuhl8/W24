\documentclass{beamer}

\title{Section 5}
\author{TA: Dante Buhl}
\institute{UCSC Math-19B}
%\date{Week 2}
\graphicspath{ {./images/} }
%\usepackage{xcolor}
\usepackage{verbatim}
\usetheme{Berkeley}
\usecolortheme{spruce}
\usefonttheme{serif}
%\usepackage[x11names]{xcolor}
\usepackage{amsmath}
\usepackage{pifont}


\begin{document}

\newcommand{\bmp}[1]{\begin{minipage}{#1\textwidth}}
\newcommand{\emp}{\end{minipage}}


%Title
\frame{\titlepage}

\section{Agenda}
\begin{frame}{Plan for Today}
    Topics to Cover
    \begin{itemize}
        \item Integration by Parts
        \item Trig Substitution
    \end{itemize}
    Section Activity 5
    \begin{itemize}
        \item 4 questions
    \end{itemize}
    Upcoming Assignments
    \begin{itemize}
        \item Homework 5 (Due Fri, Feb. $16^{th}$)
        \item Project 1 (Due Tues, Feb $20^{th}$).
    \end{itemize}
\end{frame}


\section{Review}
\begin{frame}{Learning Outcomes}
    \begin{itemize}
        \item Understanding and applying the concept of Integration by Parts
        \item Understanding how to use Trig Substitution
    \end{itemize}
\end{frame}

\begin{frame}{Integration By Parts}

Integration by parts, similar to u-substition, is motivated by a rule for differentiation. Take a guess. Its the product rule!!!

Recall: 
\[
    \frac{d}{dx} \left[f(x)g(x)\right] = f'(x)g(x) + f(x)g'(x)
\]
We can therefore impose that if we are given an integral of the form, $\int f'(x)g(x)dx$.
\[
    \int f'(x)g(x)dx = f(x)g(x) - \int f(x)g'(x)dx
\]
\footnotesize{Note: people sometimes write this in an alternate form}
\normalsize
\[
    \int u dv = u v - \int v du
\]

\end{frame}
\begin{frame}{Integration by Parts}

An example application of this practice can be seen here. 
    \[
        \int \ln(x)dx
    \]
    \[
        f'(x) = 1, \quad g(x) = \ln(x)
    \]
    \[
        f(x) = x, \quad g'(x) = \frac{1}{x}
    \]
    \[
        \int \ln(x)dx = x\ln(x) - \int 1dx 
    \]
    \[
        = x\ln(x) - x + C
    \]
\end{frame}

\begin{frame}{Now do a couple problems yourself}

    \bmp{.45}
        \[
            \int \cos^2(x)dx
        \]
        \footnotesize{Note: this can be done with the trig identity}
        \[
            \cos^2(x) + \sin^2(x) = 1
        \]
        \normalsize
    \emp
    \bmp{.45}
        \[
            \int x\ln(x)dx
        \]
    \emp

\end{frame}

\begin{frame}{Trigonometric Integrals}

Some weird trig integrals have reduction formulas obtained by Integration by Parts. An example of one is below. 
\scriptsize
    \[
        \int \sin^m(x)\cos^n(x)dx = \frac{\sin^{m+1}(x)\cos^{n-1}(x)}{n} + \frac{n - 1}{n}\int \sin^{m}(x)\cos^{n-2}(x)dx
    \]
These can be truly ugly. Hint hint, wink wink, you might be asked to prove one or two on your homework :)
\normalsize

\end{frame}


\section{Attenance}
\begin{frame}{Discussion Section Activity 5}
    \raggedright
    Woah look, the TA is about to write the code on the board!
\end{frame}



\end{document}
