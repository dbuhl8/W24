\documentclass{beamer}

\title{Section 6}
\author{TA: Dante Buhl}
\institute{UCSC Math-19B}
%\date{Week 2}
\graphicspath{ {./images/} }
%\usepackage{xcolor}
\usepackage{verbatim}
\usetheme{Berkeley}
\usecolortheme{spruce}
\usefonttheme{serif}
%\usepackage[x11names]{xcolor}
\usepackage{amsmath}
\usepackage{pifont}


\begin{document}

\newcommand{\bmp}[1]{\begin{minipage}{#1\textwidth}}
\newcommand{\emp}{\end{minipage}}


%Title
\frame{\titlepage}

\section{Agenda}
\begin{frame}{Plan for Today}
    Topics to Cover
    \begin{itemize}
        \item Trig Substitution
        \item Partial Fraction Decomposition
    \end{itemize}
    Section Activity 6
    \begin{itemize}
        \item 1 question
    \end{itemize}
    Upcoming Assignments
    \begin{itemize}
        \item Homework 6 (Due Fri, Feb. $16^{th}$)
        \item Midterm (On Mon, Feb $26^{th}$)
    \end{itemize}
\end{frame}


\section{Review}
\begin{frame}{Learning Outcomes}
    \begin{itemize}
        \item Revisiting the notion of substitution and trigonometric substitution.
        \item Applying the methods of Partial Fraction Decomposition.
    \end{itemize}
\end{frame}

\begin{frame}{Trig Substitution}

This substitution method is used when a trigonometric identity can reduce the complexity of an integral. Lets look at an example. 
\bmp{.45}
\[
    \int \sqrt{1 - x^2}dx 
\]
\footnotesize
\[
    x = \sin(\theta), \quad dx = \cos(\theta)d\theta
\]
\normalsize
\[
    \int \sqrt{\cos^2(\theta)}\cos(\theta)d\theta
\]
\[
    = \int \cos^2(\theta)d\theta
\]
\emp
\bmp{.45}
\centering 
\underline{The Usual Suspects:}
\[
    \cos^2(x) + \sin^2(x) = 1
\]
\[
    1 + \tan^2(x) = \sec^2(x)   
\]
\[
    \cot^2(x) + 1 = \csc^2(x)
\]

\emp
\end{frame}

\begin{frame}{Partial Fraction Decomposition}

Partial Fraction Decomposition expands rational function into a sum/difference of smaller rational functions. i.e. 
\[
    \frac{1}{x^2 - 5x + 6} = \frac{1}{(x-2)(x-3)} = \frac{A}{x-2} + \frac{B}{x-3}
\]
Notice that the two terms on the right are easier to integrate than the one on the left. They integrate into natural logarithms.

\end{frame}
\begin{frame}{Partial Fraction Decomposition (Continued)}

Partial Fraction Decomposition coefficients are found using a series of equations. 
\[
    1 = A(x - 3) + B(x - 2)
\]
Note: it helps to isolate the equations by orders of x. One equation for $x^0$, one for $x^1$, one for $x^2$, and so on as needed. 
\[
    A + B = 0, \quad -3A - 2B = 1
\]
\[
    A = -B, \quad -3A + 2A = 1
\]
\[
    A = -1, \quad B = 1
\]
\end{frame}

\begin{frame}{Warm Up - Partial Fraction Decomposition}
    \Large
    \[
        \int \frac{1}{x^2 - 2x}dx
    \] 
    \normalsize
    When your group is done working on this problem, review polyonomial long division and then let me know (I'll come give you the section activity code). 

\end{frame}

\end{document}
