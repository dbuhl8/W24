\documentclass{beamer}

\title{Section 7}
\author{TA: Dante Buhl}
\institute{UCSC Math-19B}
%\date{Week 2}
\graphicspath{ {./images/} }
%\usepackage{xcolor}
\usepackage{verbatim}
\usetheme{Berkeley}
\usecolortheme{spruce}
\usefonttheme{serif}
%\usepackage[x11names]{xcolor}
\usepackage{amsmath}
\usepackage{pifont}


\begin{document}

\newcommand{\bmp}[1]{\begin{minipage}{#1\textwidth}}
\newcommand{\emp}{\end{minipage}}


%Title
\frame{\titlepage}

\section{Agenda}
\begin{frame}{Plan for Today}
    Topics to Cover
    \begin{itemize}
        \item Improper Integrals
        \item Arc Length
        \item Surface Area
    \end{itemize}
    Section Activity 7
    \begin{itemize}
        \item 1 question
    \end{itemize}
    Upcoming Assignments
    \begin{itemize}
        \item Homework 7 (Due Mon, Mar. $4^{th}$)
        \item Project 2 (Will be released by the end of the week)
    \end{itemize}
\end{frame}


\section{Review}
\begin{frame}{Learning Outcomes}
    \begin{itemize}
        \item Accomodating improper integrals with the use of limits. 
        \item Integral forms for Arc Length and Surface Area
    \end{itemize}
\end{frame}

\begin{frame}{Improper Integrals}

Improper Integrals are integral cases where we find infinity arrive in some form in our work, often in the limits of integration.
\[
    \int_0^{\infty} f(x) dx = F(\infty) - F(0)
\]
What is to do be done here is to take the limit of the terms that involve infinity and see if they converge to a value or not. Take for example, $f(x) = -e^{-x}$
\[
    \int_0^{\infty} e^{-x}dx = -e^{-x}\Big|_0^{\infty} = \left(\lim_{x\to \infty} -e^{-x}\right) + 1 = 1
\]


\end{frame}

\begin{frame}{Arc Length and Surface Area}

We now introduce two new integral forms. These are integrals for 1D functions which have geometrical interpretations. 

    Arc Length (of $f(x)$ over the interval $[a, b]$)
    \[
        \int_a^b \sqrt{1 + (f'(x))^2}dx
    \]
    Surface Area (by rotating $f(x)$ about the x-axis on $[a,b]$)
    \[
        2\pi\int_a^b f(x)\sqrt{1 + (f'(x))^2}dx
    \]

\end{frame}


\end{document}
