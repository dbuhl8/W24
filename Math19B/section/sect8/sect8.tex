\documentclass{beamer}

\title{Section 8}
\author{TA: Dante Buhl}
\institute{UCSC Math-19B}
\graphicspath{ {./images/} }
\usepackage{verbatim}
\usetheme{Antibes}
\usecolortheme{beaver}
\usefonttheme{serif}
\usepackage{amsmath}
\usepackage{pifont}


\begin{document}

\newcommand{\bmp}[1]{\begin{minipage}{#1\textwidth}}
\newcommand{\emp}{\end{minipage}}


%Title
\frame{\titlepage}

\section{Agenda}
\begin{frame}{Plan for Today}
    Topics to Cover
    \begin{itemize}
        \item Infinite Sequences and Series
        \item Convergence
        \item Taylor Polynomials
    \end{itemize}
    Section Activity 8
    \begin{itemize}
        \item 5 question
    \end{itemize}
    Upcoming Assignments
    \begin{itemize}
        \item Homework 8 (Due Mon, Mar. $8^{th}$)
        \item Project 2 (Due Fri, Mar. $15^{th}$)
    \end{itemize}
\end{frame}


\section{Review}
\begin{frame}{Learning Outcomes}
    \begin{itemize}
        \item Understanding the notion of sequences and series
        \item Understanding what different convergence tests imply about a sequence or series
        \item Applying the concept of Taylor Polynomials to complex functions. 
    \end{itemize}
\end{frame}

\begin{frame}{Sequences and Series}

Sequences are collections of terms which are identified by an index. For example, the following sequence $S$ is a collection of terms $a_i$ identified by the index in the subscript, $i$.
\[
    S = \{ a_1, \cdots, a_n \}, a_i = f(n)
\]

An infinite series, is an infinite sum of terms. We can take for instance the sum of a sequence. 
\[
    \text{Infinite Series } = \sum_{i = 1}^\infty a_i 
\]

\end{frame}

\begin{frame}{Convergence}

For many sequences and series, the notion of convergence is often relevant or interesting. For example as we take $N$ to infinity, is $a_N$ finite? Does the sum $\sum_{i=1}^N a_n$ approach a value? To answer these questions we use some analytical tools:
\begin{enumerate}
    \item Integral Test
    \item Comparison Test
    \item Limit Comparison Test
    \item Ratio Test
\end{enumerate}
\end{frame}

\begin{frame}{Convergence Tests}

Integral Test:
\[
    a_n = f(n), \quad \sum_{i=1}^{\infty} a_i \le \int_1^{\infty} f(x)dx
\]
Comparison Test:
\[
    0 \le a_n \le b_n, \forall n > 0, \quad \sum_{i=1}^{\infty} a_i \le \sum_{i=1}^{\infty} b_i
\]
\[
    \text{if } \sum_{i=1}^{\infty} b_i < \infty \implies, \sum_{i=1}^{\infty} a_i < \infty
\]
\[
    \text{if } \sum_{i=1}^{\infty} a_i \text{ diverges } \implies \sum_{i=1}^{\infty} b_i \text{ diverges }
\]

\end{frame}

\begin{frame}{Convergence Tests}

Ratio Test:
\[
    \rho = \lim_{n \to \infty} = \left| \frac{a_{n+1}}{a_n}\right|
\]
\[
    \text{if } \rho < 1, \implies \sum a_n < \infty
\]
\[
    \text{if } \rho > 1, \implies \sum a_n  \text{ diverges}
\]
\[
    \text{if } \rho = 1, \text{ inconclusive }
\]

\end{frame}

\begin{frame}{Taylor Polynomials}

Taylor Polynomials (also known as Taylor Expansions) are very useful mathematical tools when you need to simplify a complex term in a local area. Say we want to fit a polynomial to $\sin(x)$ at $x = 1$. We use taylor expansion. 

\[
    T_n(x) = \sum_{i=0}^n \frac{(x - a)^i}{i!} f^{(i)}(a)
\]
\[
    T_n(x) = f(a) + (x - a)f'(a) + \frac{f''(a)}{2!}(x - a)^2 + \frac{f'''(a)}{3!}(x - a)^3 + \cdots
\]
\[
    T_3(x) = \sin(1) + (x - 1)\cos(1) - \frac{\sin(1)}{2}(x - 1)^2 - \frac{\cos(1)}{3!}(x - 1)^3
\]
\end{frame}


\end{document}
